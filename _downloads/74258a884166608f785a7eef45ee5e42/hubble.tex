\documentclass[10pt,twocolumn]{article}

\usepackage{graphicx}
\usepackage{natbib}

\title{Confirming the Hubble-Lema\^itre universe expansion law}
\author{The Data Stewards}

\begin{document}

\maketitle

\section{Introduction}

Almost a century ago, \citet{Hubble1929} published a proceedings
paper suggesting that the Universe is not static, but expanding.
Using a sample of 24 nearby galaxies, for which distance and 
velocity measurements were available, he showed that the velocity
of a galaxy is linearly related to the distance. In other words,
the larger the distance to a galaxy is, the faster is its velocity
away from us. These measurements lead to the famous Hubble-Lema\^itre 
universe expansion law, which changed our view of the universe.

The Hubble-Lema\^itre expansion law states that the velocity ($v$)
of a galaxy is linearly related to the distance between us and the
galaxy ($D$):
\begin{equation}
v = H_0 D
\end{equation}
The Hubble constant ($H_0$) describes the expansion rate of the 
universe in units of km/s/Mpc.

\section{Data analysis}

To celebrate this amazing discovery, we repeat Hubble's experiment
using modern distance and velocity data for the galaxies Hubble 
selected. We obtain these data from the SIMBAD\footnote{http://simbad.cds.unistra.fr/simbad/}
astronomical database using the Astroquery\footnote{https://astroquery.readthedocs.io/en/latest/simbad/simbad.html}
Python package. The list of objects, their distances and velocities 
are listed in Table~\ref{tab:galaxies}. 

\begin{table}
\caption{The modern velocity and distance data for the Hubble 
sample of galaxies.}
\label{tab:galaxies}
% Include a table with the object names, distances and velocities from SIMBAD.
% See how you write an astropy table to a latex table here:
% https://docs.astropy.org/en/stable/table/io.html
\end{table}

\section{Results}

We fit the velocities with a linear function using the \verb+numpy.linalg.lstsq+
method. The result is shown in Figure~\ref{fig:hubblefit}. We find that the 
Hubble constant is \textit{XXX} km/s/Mpc. Using this number, we can estimate that the 
universe is \textit{XXX} billion years old.

\begin{figure}

\caption{Plot of the velocities of the galaxies versus their distance. The
red line indicates the linear fit with the Hubble equation.}
\label{fig:hubblefit}
\end{figure}


\section{Conclusion}

% Write a couple of sentences about your result. Is the Hubble constant that
% you find consistent with what modern dedicated observatories, like Planck, find? 


\section{Data availability}

% Write this section when you reach the reproduction package stage of the workshop.


\bibliographystyle{dinat} 
\bibliography{hubble} % references are located in hubble.bib


\end{document}
